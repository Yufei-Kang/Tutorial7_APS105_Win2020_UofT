% --------------------------------------------------------------
% This is all preamble stuff that you don't have to worry about.
% Head down to where it says "Start here"
% --------------------------------------------------------------
 
\documentclass[12pt]{article}

\usepackage[margin=1in]{geometry} 
\usepackage{amsmath,amsthm,amssymb}
%\usepackage[spanish]{babel} %Castellanización
\usepackage[T1]{fontenc} %escribe lo del teclado
\usepackage[utf8]{inputenc} %Reconoce algunos símbolos
\usepackage{lmodern} %optimiza algunas fuentes
\usepackage{graphicx}
\graphicspath{ {images/} }
\usepackage{hyperref} % Uso de links
 \usepackage[most]{tcolorbox}
 \usepackage{lmodern}
\newcommand{\N}{\mathbb{N}}
\newcommand{\Z}{\mathbb{Z}}
 \usepackage{fancyhdr}
\usepackage{listings}
\usepackage{xcolor}
\usepackage{parskip} 
\usepackage{mathpazo} 

\definecolor{codegreen}{rgb}{0,0.6,0}
\definecolor{codegray}{rgb}{0.5,0.5,0.5}
\definecolor{codepurple}{rgb}{0.58,0,0.82}
\definecolor{backcolour}{rgb}{0.95,0.95,0.92}

\lstdefinestyle{mystyle}{
    backgroundcolor=\color{backcolour},   
    commentstyle=\color{codegreen},
    keywordstyle=\color{magenta},
    numberstyle=\tiny\color{codegray},
    stringstyle=\color{codepurple},
    basicstyle=\ttfamily\footnotesize,
    breakatwhitespace=false,         
    breaklines=true,                 
    captionpos=b,                    
    keepspaces=true,                 
    numbers=left,                    
    numbersep=5pt,                  
    showspaces=false,                
    showstringspaces=false,
    showtabs=false,                  
    tabsize=2
}

\lstset{style=mystyle}

 
\begin{document}
 
% --------------------------------------------------------------
%                         Start here
% --------------------------------------------------------------
 
\title{APS 105: Computer Fundamentals}
\date{}
\author{Tutorial \#3\\ 
Summer 2025}

\maketitle

\section*{Problem 1: Pointer (Winter 2017  Midterm Exam, Q6)}
Consider the following code, which uses pointers. What are the values of the variables a and b after this code is executed?
\begin{lstlisting}[language=C]
	int a, b, c, d;
	int *e, *f;
	
	a = 5;
	b = 6;
	e = &c;
	f = &d;
	*e = a + b; 
	*f = *e + b;
	e = &a;
	f = &b;
	
	*e = c + d; 
	*f = a + b;
\end{lstlisting}

\textbf{Example solution:}
\begin{lstlisting}[language=C]
e is 28 and b is 34.
\end{lstlisting}
\section*{Problem 2: Pointers and Functions (Winter 2018 Midterm, Q6)}

The following function returns the higher value between c and d. Identify the potential error in the higher function and fix it.

\begin{lstlisting}[language=C]
#include <stdio.h>

int higher(int *m, int *n) {
		int isHigher;
		
		if (m >= n)
			isHigher = m;
		else
			isHigher = n;
			
		return &isHigher;
	 }
 
int main(void) {
		int c = 9, d = 8;
		int isHigher;
		
		isHigher = higher(&c, &d); 
		printf("%d\n", isHigher);
		
		return 0;
	}
\end{lstlisting}

\textbf{Example solution:}
\begin{lstlisting}[language=C]
int higher(int *m, int *n) { 
	int isHigher;
	if (*m >= *n) 
		isHigher = *m;
	else
		isHigher = *n; 
		
	return isHigher;
}
\end{lstlisting}

\section*{Problem 3: Conversion with Functions (Winter 2019 Midterm, Q9)}
There are $0.3048$ metres in a foot, $100$ centimetres in a metre, and $12$ inches in a foot. Write a program that will accept, as input, a length in feet and inches. You do not have to check for valid input – assume the user enters positive, non-fractional values for the feet and inches. The program will output the equivalent length in metres and centimetres (rounded to the nearest centimetre).

Your code should include four functions: one for input, one for output, one to perform the calculation, and main. The function prototypes are below. For full marks, your code should not use any global variables.

\begin{lstlisting}[language=C]
void getInput(int *outFeet, int *outInches);
void printOutput(int feet, int inches, int metres, int centimetres); 
void convert(int feet, int inches, int *outMetres, int *outCentimetres);
\end{lstlisting}

\textbf{Example solution:}
\begin{lstlisting}[language=C]
#include <math.h> 
#include <stdio.h>

void getInput(int *outFeet , int *outInches) {
	printf("Please enter the feet and inches to convert: "); 
	scanf("%d %d", outFeet, outInches);
}

void printOutput(int feet, int inches, int metres, int centimetres) {
	
printf("%d feet %d inches is %d metres and %d centimetres.\n", feet , inches , metres , centimetres );
}

void convert(int feet, int inches, int *outMetres, int *outCentimetres) { 
	double length = feet + (inches / 12.0);
	double metres = length * 0.3048;
	*outMetres = metres; // truncate to integer *outCentimetres = rint((metres - *outMetres) * 100);
}

int main(void) {
	int feet, inches;
	getInput(&feet, &inches);
	int metres, centimetres;
	convert(feet, inches, &metres, &centimetres); printOutput(feet, inches, metres, centimetres); 
	return 0;
}
\end{lstlisting}

\section*{Problem 4: Print Pattern (Modified from Winter 2022 Midterm, Q3)}

Write a complete C program that first prompts the user to enter the number of rows in a pattern, then prints a pattern were the first and last row and column is filled with stars and a diagonal in the square is drawn. The following are sample outputs when number of rows is $5$ and $8$:

\begin{tcolorbox}[colback=gray!10, boxrule=0pt, sharp corners, enhanced jigsaw, left=3mm, right=3mm, top=1mm, bottom=1mm]
	Enter number of rows: 5\\
	*****\\
	* \ \ **\\
	* \  *\   *\\
	** \ \  * \\
	*****\\
\end{tcolorbox}

\begin{tcolorbox}[colback=gray!10, boxrule=0pt, sharp corners, enhanced jigsaw, left=3mm, right=3mm, top=1mm, bottom=1mm]
	Enter number of rows: 8\\
	********\\
	*\phantom{*}\phantom{*}\phantom{*}\phantom{*}\phantom{*}**\\
	*\phantom{*}\phantom{*}\phantom{*}\phantom{*}*\phantom{*}*\\
	*\phantom{*}\phantom{*}\phantom{*}*\phantom{*}\phantom{*}* \\
	*\phantom{*}\phantom{*}*\phantom{*}\phantom{*}\phantom{*}*\\
	*\phantom{*}*\phantom{*}\phantom{*}\phantom{*}\phantom{*}*\\
	**\phantom{*}\phantom{*}\phantom{*}\phantom{*}\phantom{*}*\\
	********\\
\end{tcolorbox}

\textbf{Example solution:}
\begin{lstlisting}[language=C]
#include <stdio.h> 

int main(void) {
	int n;
	printf("Enter number of rows: "); 
	scanf("%d", &n);
	
	for (int row = 1; row <= n; row++) {
		for (int col = 1; col <= n; col++) {
		if (col == 1 || row == 1 || col + row - 1 == n || row == n || col ==n) {
			printf("*"); 
		} 
		else {
			printf(" ");
		}
	}
		printf("\n"); 
	}
	return 0; 
}
\end{lstlisting}


\end{document}