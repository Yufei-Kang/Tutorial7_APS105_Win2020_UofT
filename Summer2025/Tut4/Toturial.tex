% --------------------------------------------------------------
% This is all preamble stuff that you don't have to worry about.
% Head down to where it says "Start here"
% --------------------------------------------------------------
 
\documentclass[12pt]{article}

\usepackage[margin=1in]{geometry} 
\usepackage{amsmath,amsthm,amssymb}
%\usepackage[spanish]{babel} %Castellanización
\usepackage[T1]{fontenc} %escribe lo del teclado
\usepackage[utf8]{inputenc} %Reconoce algunos símbolos
\usepackage{lmodern} %optimiza algunas fuentes
\usepackage{graphicx}
\graphicspath{ {images/} }
\usepackage{hyperref} % Uso de links
 \usepackage[most]{tcolorbox}
 \usepackage{lmodern}
\newcommand{\N}{\mathbb{N}}
\newcommand{\Z}{\mathbb{Z}}
 \usepackage{fancyhdr}
\usepackage{listings}
\usepackage{xcolor}
\usepackage{parskip} 
\usepackage{mathpazo} 

\definecolor{codegreen}{rgb}{0,0.6,0}
\definecolor{codegray}{rgb}{0.5,0.5,0.5}
\definecolor{codepurple}{rgb}{0.58,0,0.82}
\definecolor{backcolour}{rgb}{0.95,0.95,0.92}

\lstdefinestyle{mystyle}{
    backgroundcolor=\color{backcolour},   
    commentstyle=\color{codegreen},
    keywordstyle=\color{magenta},
    numberstyle=\tiny\color{codegray},
    stringstyle=\color{codepurple},
    basicstyle=\ttfamily\footnotesize,
    breakatwhitespace=false,         
    breaklines=true,                 
    captionpos=b,                    
    keepspaces=true,                 
    numbers=left,                    
    numbersep=5pt,                  
    showspaces=false,                
    showstringspaces=false,
    showtabs=false,                  
    tabsize=2
}

\lstset{style=mystyle}

 
\begin{document}
 
% --------------------------------------------------------------
%                         Start here
% --------------------------------------------------------------
 
\title{APS 105: Computer Fundamentals}
\date{}
\author{Tutorial \#4\\ 
Summer 2025}

\maketitle

\section*{Problem 1: Simple Array (Winter 2024  Final Exam, Q1)}
Write a single C statement that declares a $1$D array of $100$ integers named \textit{arr}, where the first three elements are set to $1$, $3$ and $10$ and the remaining $97$ elements are set to $0$.

\iffalse
\textbf{Example solution:}
\begin{lstlisting}[language=C]
int arr[100] = {1, 3, 10};
\end{lstlisting}
\fi
\section*{Problem 2: Dynamic Allocation of Arrays (Fall 2015 Final Exam, Q2)}
Write one C statement that use \textit{malloc} to dynamically allocate an array of elements of type \textit{double}. The allocated array should be called \textit{list}.
\iffalse
\textbf{Example solution:}
\begin{lstlisting}[language=C]
double *list = (double *)malloc(1000 * sizeof(double));
\end{lstlisting}
\fi
\section*{Problem 3: 2D Arrays and Functions}

Write a C function called \textit{borderSum} that adds all the border elements of the top-left $2$D square matrix inside a $2$D square array. For example, for the array below, if $n$ is $3$, we should add ${1, 2, 3, 7, 11, 10, 9, 5}$ and return 48. If n is 0, the function should return $0$. If n is $1$, the function should return the top-left element, which is $1$ for the array below.

\[
\begin{array}{cccc}
1 & 2 & 3 &4\\
5 & 6 & 7&8\\
9&10&11&12\\
13&14&15&16\\
\end{array}
\]

The header of the \textit{borderSum} function is provided below, where $26$ is the number of rows and columns in the $2$D array mat, and n is the size of the square matrix to which we need to get the sum of its border. You can safely assume $n \leq 26$.

\begin{lstlisting}[language=C]
int borderSum(int mat[][26], int n){
\end{lstlisting}
\iffalse
\textbf{Example solution:}
\begin{lstlisting}[language=C]
int borderSum(int mat[][26], int n) { 
	int sum = 0, row, col;
	
	if (n == 0) {
		return 0;
	} 	else if (n == 1) {
			sum = mat [0][0];
	} 	else if (n == 2) { //unnecessary else-if, but fine if it is there 
			sum = mat [0][0] + mat [0][1] + mat [1][0] + mat [1][1];
	} 	else {
			for (col = 0; col < n; col++) {
			sum += mat[0][col] + mat[n - 1][col]; 
			}
	
			for (row = 1; row < n - 1; row++) {
			sum += mat[row][0] + mat[row][n - 1];
		} 
	}
	return sum; 
}
\end{lstlisting}
\fi
\section*{Problem 4: Working with Arrays and Pointers (Winter 2024 Final Exam, Q3)}
In the box provided below, write the output generated after the following program is completely executed.

\begin{lstlisting}[language=C]
#include <stdio.h> 

int main() {
	int i = 1, *first;
	int myArray[] = {0, 1, 2, 3, 7, 11, 15};
	
	while (myArray[i] < 3) { 
		*( myArray + i) = 3; 
		i++;
	}
	
	*( myArray + i) = 0; 
	first = myArray + i + 1; 
	myArray[0] = *first; 
	*first = myArray[i];
	
	printf("i = %d, *first = %d\n", i, *first); 
	printf("myArray has ");
	for (int i = 0; i < 7; i++) {
	printf("%d, ", myArray[i]); 
	}
	
	return 0; 
}
\end{lstlisting}
\iffalse
\textbf{Example solution:}
\begin{lstlisting}[language=C]
i = 3, *first = 0
myArray has 7, 3, 3, 0, 0, 11, 15,
\end{lstlisting}
\fi
\section*{Problem 5:  Dynamic Memory Allocation ( Winter 2023 Midterm, Q9)}

Create a code segment in C that takes a dynamically allocated array and adds another element to it. Use the following variables (assume they are already defined).
\begin{enumerate}
	\item \textit{int newNumber}, an integer value to be added into the array
	\item \textit{ int numbersInArray}, a non-negative value of how many elements are presently in the array.
	\item \textit{int *pArray}, a pointer to the array which is in dynamically allocated memory. The array already exists and has one or more elements. The code segment must add \textit{newNumber} to the present array, leaving \textit{pArray} pointing to the new array with \textit{newNumber} in it. Since the present array size is not large enough, you will have to allocate a new space and move the present values plus \textit{newNumber} to this space. Avoid memory leaks.
\end{enumerate}
\iffalse
\textbf{Example solution:}
\begin{lstlisting}[language=C]
int *extArray = (int *)malloc(sizeof(int) * (numInArray + 1));
extArray[0] = newNum; // add at first since easier

for (int i = 1; i <= numInArray; i++){
	extArray[i] = pArray[i - 1]; // transfer old stuff
}

free(pArray);
pArray = extArray; // reassign pointer to new array space
\end{lstlisting}
\fi
\end{document}